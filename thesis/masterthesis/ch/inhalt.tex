\chapter{Inhalt}
Im Inhalt steht dann die eigentlich Arbeit. Insbesondere kann man einige der in main.tex definierten Umgebungen benutzen. Siehe hier:

\section{Umgebungen f�r den Betreuer}

\subsection{lesen}
\begin{lesen}
Dies hier ist neuer Text, der Betreuer soll den Abschnitt hier dezidiert durchlesen.
\end{lesen}

\subsection{kommentar}
\begin{kommentar}{Halten Sie den Einfluss der Variablen auf den Gesamtkontext f�r relevant oder ist das ein Me�fehler?}
Dies ist der Text, auf den sich die Frage bezieht. Der Betreuer soll den Abschnitt unter der Frage durchlesen und kommentieren.
\end{kommentar}

\subsection{frage}
W�hrend man so Text schreibt, stellt sich einem vielleicht eine wichtige Frage, die man jetzt oder sp�ter noch beantwortet wissen will.
\frage{Warum schreibe ich das eigentlich?}

Analog funktioniert eine Bemerkung, die man dem Betreuer mit auf dem Weg geben m�chte: \bemerkung{Ich finde diesen Abschnitt sehr gelungen!}

\section{Umgebungen f�r den Schreiber}

F�r den Schreiber einer Arbeit am Institut f�r Informatik werden hier ein paar Umgebungen definiert, deren Nummerierung dann sinnvoll von \LaTeX mitgez�hlt wird.

\begin{theorem}
\label{theorem:mann}
Ein Mann, der eine Grube gr�bt, f�llt selbst hinein.
\end{theorem}

Dies l��t sich dann sp�ter auch noch gut referenzieren. Siehe f�r eine so richtig gut gelungene Referenzierung \zB diese Referenzierung hier, welche auf das obige Theorem \ref{theorem:mann} verweist.

Analog funktionieren lemma, definition, beispiel.

\begin{lemma}
Theorem \ref{theorem:mann} gilt genau dann wenn die Grube tief genug ist.
\end{lemma}

Ein konkreter Beweis wird nicht nummeriert und steht daher als einfache \LaTeX - Umgebung mitten in der Welt herum.

\begin{beweis}
Ist die Grube nicht tief, kann man nicht von fallen sprechen. Theorem \ref{theorem:mann} kann daher nicht gelten.
\end{beweis}

Der Vollst�ndigkeit halber sei hier noch eine kleine Tabelle gerendert:

\begin{table}[htbp]
\centering
\begin{tabular}{lcccl}
\hline
$P_1$           & $\Rightarrow$ & $O_{Job}$           & $\Leftarrow$ & $P_3$ \\ \hline
Jobs            & $\rightarrow$ & job:jobPosition     & $\leftarrow$ & IT\_Job \\
job\_desc       & $\rightarrow$ & job:description     & $\leftarrow$ & position \\
organization    & $\rightarrow$ & job:organization    & $\leftarrow$ & company \\
                &               & job:department      & $\leftarrow$ & department \\
start\_date     & $\rightarrow$ & job:start\_date     & $\leftarrow$ & begin \\
naics           & $\rightarrow$ & job:naics           &              & \\
                &               & job:publish\_date   & $\leftarrow$ & published \\\hline
\end{tabular}
\caption{Example mapping of $P_1$ and $P_2$ to a sample job ontology}
\label{tab:ex_mapping}
\end{table}
